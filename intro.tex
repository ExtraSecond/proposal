\section{Introduction}

\todo{1-2 paragraphs about what problem you are solving and why it is important.}

Internet of Things(IoT) has posted new security threats since its first appearance. 
It is estimated that "by 2020, More Than 25 Percent of Identified Attacks in Enterprises Will Involve IoT"\cite{gartner2016gartner}.

\todo{1-2 paragraphs about how other people have solved this problem and how they fall short.}

There have been efforts on how to secure IoT devices from a system perspective like \textit{The Seven Properties}\cite{hunt2017the} proposed by Galen Hunt.
People have also worked on how to create new systems including SPIN\cite{hesselman2017spin} and Tock\cite{levy2017tock}.

\todo{1 paragraph (optional) what is difficult about this problem.}

\todo{1-2 paragraphs about your plan on why it is better than current approaches.}

In this work, we present the Rust IoT Operating System(RITOS).
We plan to write a new IoT OS from scratch using Rust in the first stage.
Rust is a programming language designed to do system programming and has safety mechanisms designed in the core language.
It has powerful compile-time checks that find most of the common memory bugs.
Therefore we can guarantee that there is a minimum number of bugs as most of them can be opt-out at compile time. 
Another advantage of our customized OS is that we can reduce the trusted computing base(TCB) to a small LOC.

In the second stage, we would re-organize the architecture of the devices.
We believe that \textit{You can't corrupt invisible things}.
Therefore we arranged devices so that most of the \textit{Slave devices} are invisible from the perspective of the public network.
Machines outside this IoT network can only talk to the \textit{Master device}.
We added extra authentication before a user can talk to the master device.

\todo{1 paragraph to summarize results, or for your project proposals summarize your anticipated results.}

In the end, we would build this network. 
We would use 2 Raspberry Pis to mimic a master device and a slave device. 
We as the user would use a shell from our Laptop to issue commands(Since we are not interested in writing a fancy UI).